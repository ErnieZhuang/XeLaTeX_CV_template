%%%%%%%%%%%%%%%%%%%%%%%%%%%%%%%%%%%%%%%%%
% 模板資訊:
% 模板名稱:莊程翔 - CV
% 版本:1.0 (2023.03.21)
% 修改者:莊程翔 Ernie Cheng-Xiang Zhuang
% 編譯器:XeLaTeX
%
% 原始模板的資訊:
% 模板名稱:Wilson Resume/CV
% 作者:1. Howard Wilson (https://github.com/watsonbox/cv_template_2004)
%      2. Vel (vel@latextemplates.com)
% 編譯器:XeLaTeX
% 授權:CC BY-NC-SA 3.0 (http://creativecommons.org/licenses/by-nc-sa/3.0/)
% 下載連結:http://www.latextemplates.com/template/wilson-resume-cv
%
% 製作本模板之目的:
% 為了讓 LaTeX 初學者能夠毫不費力地寫出整潔專業的 CV,因此我針對 Howard Wilson 和 Vel 製作的 CV 做了大幅度的修改及附上清楚明瞭的註解。
% 如果您有任何問題,可以透過以下兩種方式聯繫我: 
% 1. 網站:https://www.ernie-zhuang.com/contact
% 2. Email:erniezhuang1127@gmail.com
%%%%%%%%%%%%%%%%%%%%%%%%%%%%%%%%%%%%%%%%%

%----------------------------------------------------------------------------------------
%	封包與文檔配置
%----------------------------------------------------------------------------------------

\documentclass[utf8, 12pt]{article} % 設定字體大小及 UTF-8 編碼

%%%%%%%%%%%%%%%%%%%%%%%%%%%%%%%%%%%%%%%%%
% 模板資訊:
% 模板名稱:莊程翔 - CV
% 版本:1.0 (2023.03.21)
% 修改者:莊程翔 Ernie Cheng-Xiang Zhuang
% 編譯器:XeLaTeX
%
% 原始模板的資訊:
% 模板名稱:Wilson Resume/CV
% 作者:1. Howard Wilson (https://github.com/watsonbox/cv_template_2004)
%      2. Vel (vel@latextemplates.com)
% 編譯器:XeLaTeX
% 授權:CC BY-NC-SA 3.0 (http://creativecommons.org/licenses/by-nc-sa/3.0/)
% 下載連結:http://www.latextemplates.com/template/wilson-resume-cv
%
% 製作本模板之目的:
% 為了讓 LaTeX 初學者能夠毫不費力地寫出整潔專業的 CV,因此我針對 Howard Wilson 和 Vel 製作的 CV 做了大幅度的修改及附上清楚明瞭的註解。
% 如果您有任何問題,可以透過以下兩種方式聯繫我: 
% 1. 網站:https://www.ernie-zhuang.com/contact
% 2. Email:erniezhuang1127@gmail.com
%%%%%%%%%%%%%%%%%%%%%%%%%%%%%%%%%%%%%%%%%

%----------------------------------------------------------------------------------------
%	封包與文檔配置
%----------------------------------------------------------------------------------------

% 設定為 A4 紙的大小及上下左右的邊界
\usepackage[a4paper, left=3.18cm, right=3.18cm, top=2.54cm, bottom=2.54cm]{geometry} 

% 超連結的封包
\usepackage[colorlinks, bookmarks = false]{hyperref}

%% 設定各種超連結的顏色
\hypersetup{
	linkcolor = red,
	citecolor = blue,
	filecolor = blue,
	urlcolor = blue
} 

% 禁止節(section)編號
\setcounter{secnumdepth}{0} 

% 輸出的字體是以 Type 1 編碼
\usepackage[T1]{fontenc} 

% 自訂字體的封包
\usepackage{fontspec} 

%% 設定英文字體
\setmainfont[Path = ./fonts/,
Extension = .otf,
BoldFont = Erewhon-Bold,
ItalicFont = Erewhon-Italic,
BoldItalicFont = Erewhon-BoldItalic,
SmallCapsFeatures = {Letters = SmallCaps}
]{Erewhon-Regular}

%% 設定中文字體
\usepackage{xeCJK}
\setCJKmainfont[AutoFakeBold=3.5 , AutoFakeSlant=0.2, Path = fonts/]{源雲明體.ttf}
\setCJKmonofont[AutoFakeBold=3.5 , AutoFakeSlant=0.2, Path = fonts/]{源雲明體.ttf}
\XeTeXlinebreaklocale "zh"
\XeTeXlinebreakskip = 0pt plus 1pt

% 自訂字體顏色的封包
\usepackage{color} 

%% 自訂一個顏色
\definecolor{Myblue}{RGB}{57,108,144}

% 允許自行定義標題的封包
\usepackage{sectsty}

%% 設定節(section)為自訂的顏色
\sectionfont{\color{Myblue}} 

% 設定每段不向內縮進
\setlength\parindent{0pt} 

% 設定不向內縮進的標籤
\newenvironment{itemize-noindent}
{\setlength{\leftmargini}{0em}\begin{itemize}}
{\end{itemize}}

% 自訂方形的項目標籤
\newcommand{\sqbullet}{~\vrule height 1ex width .8ex depth -.2ex} 

% 註釋掉大部分的封包
\usepackage{comment}

%----------------------------------------------------------------------------------------
%	頁首
%----------------------------------------------------------------------------------------

\renewcommand{\title}[1]{
{\huge{\color{Myblue}\textbf{#1}}}\\ % Header section name and color
\rule{\textwidth}{0.5mm}\\ % Rule under the header
}

%----------------------------------------------------------------------------------------
%	研究志趣
%-----------------------------------------------------------------------------------------

\newcommand{\interests}[1]{
\begin{tabbing}
\hspace{5mm} \= \kill
#1
\end{tabbing}
\vspace{-10mm}
}

\newcommand{\interest}[1]{\sqbullet \> #1\\[3pt]} % Define the item name

%----------------------------------------------------------------------------------------
%	電腦技能
%----------------------------------------------------------------------------------------

\newcommand{\skills}[2]{
\begin{tabbing}
\hspace{5mm} \= \kill
\sqbullet \>\+ \textbf{#1} \\
\begin{minipage}{\textwidth}
\vspace{2mm}
#2
\end{minipage}
\end{tabbing}
}

%----------------------------------------------------------------------------------------
%	進行中的計畫
%----------------------------------------------------------------------------------------

\newcommand{\Progress}[1]{
\begin{tabbing}
\hspace{5mm} \= \kill
\sqbullet \>\+ 
\begin{minipage}[t]{14.14cm} % 21(width of A4) - 3.18(left & right margin)*2 - 0.5(item)
#1
\end{minipage}
\end{tabbing}
}


%----------------------------------------------------------------------------------------
%	標籤式區塊(Tabbed Block)
%----------------------------------------------------------------------------------------

\newcommand{\tabbedblock}[2]{
\begin{tabbing}
\hspace{3.64cm} \= \kill
{\bf#1}\> 
\begin{minipage}[t]{11cm} % 21(width of A4) - 3.18(left & right margin)*2 - 3.92(time box)
#2
\end{minipage}
\end{tabbing}
}

%----------------------------------------------------------------------------------------
%	更新資訊
%----------------------------------------------------------------------------------------

\newcommand{\update}[1]
{\raggedleft
\par \vfill \noindent {\small #1 (\today)}
}
 % 載入封包與文檔配置

%----------------------------------------------------------------------------------------

\begin{document}

\fontsize{12}{20pt}\selectfont % 設定字體與間距的大小

%----------------------------------------------------------------------------------------
%	CV 標題
%----------------------------------------------------------------------------------------

\title{莊程翔 -- CV} % 標題

%------------------------------------------------

\parbox{0.5\textwidth}{ % 第一個區塊
\begin{tabbing}
\hspace{2cm} \= \hspace{4cm} \= \kill % 區塊內兩欄位的空間大小
{\bf 生日} \> 1999.11.27 \\ % 生日
{\bf 國籍} \> 臺灣 % 國籍
%\\
%{\bf 地址} \> 300 新竹市東區\\ % 地址 1
%\> 光復路二段101號  % 地址 2
\end{tabbing}}
%
\hfill % 將兩區塊水平推至頁面的左右邊界
%
\parbox{0.5\textwidth}{ % 第二個區塊
\begin{tabbing}
\hspace{2cm} \= \hspace{4cm} \= \kill % 區塊內兩欄位的空間大小
{\bf Email} \> \href{mailto:john@smith.com}{erniezhuang1127@gmail.com} \\ % Email
{\bf 網站} \> \href{https://www.ernie-zhuang.com}{https://www.ernie-zhuang.com} % 網站
%\\
%{\bf 電話} \> +886 2 345 6789 \\ % 電話
%{\bf 手機} \> +886 123 456 789 \\ % 手機
\end{tabbing}}

%----------------------------------------------------------------------------------------
%	個人資料
%----------------------------------------------------------------------------------------

\section{個人資料}

我目前在國立清華大學攻讀經濟學碩士學位,
並於2022年6月從國立高雄大學應用經濟學系取得學士學位。
在大學期間,
我對個體經濟學理論和計量經濟學產生了濃厚的興趣,
因此我決定深入研究這些領域。
而目前我專注於研究國際寡佔市場下的貿易和產業政策。
另外,在我空閒時,
我喜歡閱讀經濟學、數學、統計學和資料科學的相關書籍。\\

在我大學就讀經濟系時,
完成了幾項研究,
讓我得以將所學的專業知識應用到實際問題上。
例如,我曾在學校擔任研究助理,
推估臺灣未來10年勞動市場上各職業的職缺。
此外,我還曾獲得國科會的大專生研究計畫之補助,
藉此探究就學貸款是否造成階級複製加劇。
這些經驗加深了我對於經濟分析複雜性的理解,
並激勵我繼續在經濟學的領域進行研究。
很開心能夠繼續探索經濟學的前沿研究議題,
並通過自己的研究為該領域做出貢獻。

%----------------------------------------------------------------------------------------
%	教育
%----------------------------------------------------------------------------------------

\section{教育}

\tabbedblock{2022.09 -- 迄\hspace{1.5em}今}
{{\bf 碩士},經濟學系,國立清華大學,臺灣。}

\tabbedblock{2020.02 -- 2022.06}
{{\bf 學士},應用經濟學系,國立高雄大學,臺灣。}

\tabbedblock{2019.09 -- 2020.01}
{{\bf 轉學},財務金融學系 \& 財務與計算數學系,義守大學,臺灣。}

\tabbedblock{2018.09 -- 2019.06}
{{\bf 轉學},醫學影像暨放射科學系,義守大學,臺灣。}

\tabbedblock{2015.09 -- 2018.06}
{{\bf 高中},新北市立丹鳳高級中學,臺灣。}

%----------------------------------------------------------------------------------------
%	研究志趣
%----------------------------------------------------------------------------------------

\section{研究志趣}

\interests{
\interest{產業組織理論與應用賽局理論}
\interest{國際貿易}
\interest{都市經濟學}
\interest{勞動經濟學 (特別是教育經濟學)}
\interest{應用計量經濟學與資料科學}
}

%----------------------------------------------------------------------------------------
%	電腦技能
%----------------------------------------------------------------------------------------
\section{電腦技能}

\skills{R}
{資料處理與分析,資料視覺化。}

\skills{Stata}
{資料處理與分析。}

\skills{Julia}
{數值分析。}

\skills{LaTeX}
{排版文章、簡報及書籍。}

%----------------------------------------------------------------------------------------
%	研究經驗
%----------------------------------------------------------------------------------------

\section{研究經驗}

\tabbedblock{2023.02 -- 迄\hspace{1.5em}今}
{研究助理,經濟研究所,中央研究院,臺灣。}

\tabbedblock{2021.03 -- 2022.07} 
{研究助理,應用經濟學系,國立高雄大學,臺灣。}

\tabbedblock{2020.11 -- 2022.02}
{研究助理,校務研究辦公室,國立高雄大學,臺灣。}

%----------------------------------------------------------------------------------------
%	教學經驗
%----------------------------------------------------------------------------------------

\section{教學經驗}

\tabbedblock{2023.02 -- 迄\hspace{1.5em}今}
{
教學助理,經濟學系,%
國立清華大學,個體經濟學二 (大學部)。
}

\tabbedblock{2022.09 -- 2023.01}
{
教學助理,經濟學系,國立清華大學,統計學一 (大學部) 及個體經濟學二 (大學部)。
}

%----------------------------------------------------------------------------------------
%	榮譽和獎勵
%----------------------------------------------------------------------------------------

\section{榮譽和獎勵}

\tabbedblock{2023.02}
{
傑出教學助理,教學發展中心,國立清華大學。
}

%----------------------------------------------------------------------------------------
%	進行中的計畫
%----------------------------------------------------------------------------------------

\section{進行中的計畫}

\Progress{
「人格特質、軟技能和個人行為對數學成就的影響」
}

\Progress{
「就學貸款是否造成階級複製加劇」
}

%----------------------------------------------------------------------------------------
%	推薦人
%----------------------------------------------------------------------------------------

\begin{comment}

\section{推薦人}

\parbox{0.5\textwidth}{ % 第一個推薦人的區塊
\begin{tabbing}
\hspace{2cm} \= \hspace{4cm} \= \kill % 區塊內兩欄位的空間大小
{\bf 姓名} \> 張鐘牟 \\ % 推薦人的姓名
{\bf 服務單位} \> 台灣雞體電鹿製造股份有限公司 \\ % 推薦人的服務單位
{\bf 職位} \> 執行長 \\ % 推薦人的職位
{\bf 聯絡資訊} \> \href{mailto:chicken@LaTeX.com}{chicken@LaTeX.com} % 推薦人的聯絡資訊
\end{tabbing}}
%
\hfill % 將兩區塊水平推至頁面的左右邊界
%
\parbox{0.5\textwidth}{ % 第二個推薦人的區塊
\begin{tabbing}
\hspace{2cm} \= \hspace{4cm} \= \kill % 區塊內兩欄位的空間大小
{\bf 姓名} \> 管棕明 \\ % 推薦人姓名
{\bf 服務單位} \> 第一學府大學財務金融學系 \\ % 推薦人的服務單位
{\bf 職位} \> 特聘教授 \\ % 推薦人的職位 
{\bf 聯絡資訊} \> \href{mailto:tomorrow@LaTeX.tw}{tomorrow@LaTeX.com} % 推薦人的聯絡資訊
\end{tabbing}}

\end{comment}

%----------------------------------------------------------------------------------------
%	更新資訊
%----------------------------------------------------------------------------------------

\update{莊程翔\_CV} % 輸入檔名

%----------------------------------------------------------------------------------------

\end{document}
